\documentclass{article}
\usepackage[a4paper,margin=1in]{geometry} % Adjusts margins
\usepackage{amsmath}
\usepackage{amssymb}
\usepackage{cancel}

\begin{document}

\section*{Derivations}

\subsection*{Computation of the posterior $p\left(r_{t} \mid x_{1:t+h}\right)$}

Starting with $h=1$, the joint distribution $p(r_{t},x_{1:t+1})$ is:

\begin{align}
p(r_{t},x_{1:t+1}) & =\sum_{r_{t+1}}p(r_{t},r_{t+1},x_{1:t+1})\\
 & =\sum_{r_{t+1}}p(r_{t},x_{1:t})p(r_{t+1},x_{t+1}\vert r_{t},x_{1:t})\\
 & =p(r_{t},x_{1:t})\sum_{r_{t+1}}p(x_{t+1}\vert r_{t},r_{t+1},x_{1:t})p(r_{t+1}\vert r_{t},\cancel{x_{1:t}})\\
 & =p(r_{t},x_{1:t})\sum_{r_{t+1}}p(x_{t+1}\vert r_{t},x_{t+1-r_{t+1}:t})p(r_{t+1}\vert r_{t})
\end{align}

Explanations:
\begin{itemize}
    \item $p(x_{t+1}\vert r_{t},r_{t+1},x_{1:t})=p(x_{t+1}\vert r_{t},x_{t+1-r_{t+1}:t})$, because knowing $r_{t+1}$ just selects the previous observations w.r.t. which we condition on. If $r_{t+1}=0$, this becomes $p(x_{t+1}\vert r_{t},x_{t+1-r_{t+1}:t})=p(x_{t+1})$, which is the prior for observing $x_{t+1}$.
    \item The transition probability $p(r_{t+1}\vert r_{t},x_{1:t})$ from the run length at time $t$ to the run length at time $t+1$ does not depend on the history of observations $x_{1:t}$. This is a model assumption in BOCD.
\end{itemize}

For $h=2$, the joint distribution $p(r_{t},x_{1:t+2})$ is:

\begin{align}
p(r_{t},x_{1:t+2}) & =\sum_{r_{t+1},r_{t+2}}p(r_{t},r_{t+1},r_{t+2},x_{1:t+2})\\
 & =\sum_{r_{t+1},r_{t+2}}p(r_{t},r_{t+1},x_{1:t+1})p(r_{t+2},x_{t+2}\vert r_{t},r_{t+1},x_{1:t+1})\\
 & =\sum_{r_{t+1},r_{t+2}}p(r_{t},r_{t+1},x_{1:t+1})p(x_{t+2}\vert\cancel{r_{t}},r_{t+1},r_{t+2},x_{1:t+1})p(r_{t+2}\vert\cancel{r_{t}},r_{t+1},\cancel{x_{1:t+1}})\\
 & =\sum_{r_{t+1},r_{t+2}}p(r_{t},r_{t+1},x_{1:t+1})p(x_{t+2}\vert r_{t+1},x_{t+2-r_{t+2}:t+1})p(r_{t+2}\vert r_{t+1})\\
 & =p(r_{t},x_{1:t})\sum_{r_{t+1},r_{t+2}}p(x_{t+1}\vert r_{t},x_{t+1-r_{t+1}:t})p(r_{t+1}\vert r_{t})p(x_{t+2}\vert r_{t+1},x_{t+2-r_{t+2}:t+1})p(r_{t+2}\vert r_{t+1})
\end{align}

Explanations:
\begin{itemize}
    \item The same reasoning as before applies to both, the predictive distribution as well as the transition probability.
    \item In the last step, the previous result was used.
\end{itemize}

By following this pattern, for arbitrary $h\geq0$ the joint distribution $p\left(r_{t},x_{1:t+h}\right)$ becomes:

\begin{align}
p(r_{t},x_{1:t+h})=p(r_{t},x_{1:t})\underbrace{\sum_{r_{t+1},...,r_{t+h}}\prod_{m=1}^{h}\left[p(x_{t+m}\vert r_{t+m-1},x_{t+m-r_{t+m}:t+m-1})p(r_{t+m}\vert r_{t+m-1})\right]}_{=:M}
\end{align}

The factor $M$ is understood to be one, if $h=0$.\\\\

The joint distribution $p\left(r_{t},x_{1:t+h}\right)$ can therefore be computed by the product of the joint distribution $p(r_{t},x_{1:t})$, which BOCD computes anyway, and a factor $M$, which depends on the predictive and transition probabilities at future time steps. These objects are components of the BOCD algorithm, computed at every time step and reusable internally within the implementation.

The run length posterior can be computed from the joint distribution:

\begin{align}
p(r_{t}\vert x_{1:t+h})=\frac{p(r_{t},x_{1:t+h})}{p(x_{1:t+h})}=\frac{p(r_{t},x_{1:t+h})}{\sum_{r_{t}}p(r_{t},x_{1:t+h})}
\end{align}

The factor $M$ can be best understood as being the product of $h$ matrices $M_{t+m}$. Let's, e.g., assume that $t=1$. The first matrix, for which $m=1$, can be written as:

\begin{align}
M_{2}=\underbrace{\left[\begin{array}{ccc}
p(x_{2}) & p(x_{2}\vert x_{1}) & 0\\
p(x_{2}) & 0 & p(x_{2}\vert x_{1})
\end{array}\right]}_{p(x_{2}\vert r_{1},x_{2-r_{2}:1})}\circ\underbrace{\left[\begin{array}{ccc}
p(r_{2}=0\vert r_{1}=0) & p(r_{2}=1\vert r_{1}=0) & 0\\
p(r_{2}=0\vert r_{1}=1) & 0 & p(r_{2}=2\vert r_{1}=1)
\end{array}\right]}_{p(r_{2}\vert r_{1})},
\end{align}

where $\circ$ denotes element-wise multiplication. Similarly, for $m=2$ we have:

\begin{align}M_{3} & =\underbrace{\left[\begin{array}{cccc}
p(x_{3}) & p(x_{3}\vert x_{1}) & 0 & 0\\
p(x_{3}) & 0 & p(x_{3}\vert x_{1:2}) & 0\\
p(x_{3}) & 0 & 0 & p(x_{3}\vert x_{1:2})
\end{array}\right]}_{p(x_{3}\vert r_{2},x_{3-r_{3}:2})}\\
 & \circ\underbrace{\left[\begin{array}{cccc}
p(r_{3}=0\vert r_{2}=0) & p(r_{3}=1\vert r_{2}=0) & 0 & 0\\
p(r_{3}=0\vert r_{2}=1) & 0 & p(r_{3}=2\vert r_{2}=1) & 0\\
p(r_{3}=0\vert r_{2}=2) & 0 & 0 & p(r_{3}=3\vert r_{2}=2)
\end{array}\right]}_{p(r_{3}\vert r_{2})}
\end{align}

For all matrices $M_{t+m}$, $r_{t+m-1}$ increments along the rows and $r_{t+m}$ increments along the columns, both starting at $r_{t+m-1}=r_{t+m}=0$ in the upper-left element. Elements for which neither the condition $r_{t+m}=0$ nor $r_{t+m-1}+1=r_{t+m}$ is satisfied are zero.

If $h$ was just two, then $M=M_{2}M_{3}$ in this example. The joint distribution can be written as a vector:

\begin{align}
\left[\begin{array}{c}
p(r_{1}=0,x_{1:3})\\
p(r_{1}=1,x_{1:3})
\end{array}\right]=\left[\begin{array}{c}
p(r_{1}=0,x_{1})\\
p(r_{1}=1,x_{1})
\end{array}\right]\circ M
\end{align}

It is possible to use the particular structure of each matrix $M_{t+m}$ (which has only elements in the first column and first upper diagonal) to implement this matrix multiplication efficiently (see \texttt{bocd.\_log\_matmul\_fast()}).

\end{document}